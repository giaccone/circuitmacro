\documentclass[a4paper,12pt]{article}

\usepackage{geometry}
\geometry{margin=2.5cm}
\usepackage{circuit}
\usepackage{lmodern}
\usepackage{graphicx}
\usepackage{hyperref}
\usepackage{enumitem}
\usepackage{amsmath}
\usepackage{titlesec}
\usepackage{titling}

\usepackage{listings}
\lstset{basicstyle=\ttfamily, columns=fullflexible, breaklines=true}
% Define custom color scheme
\definecolor{ltgray}{rgb}{0.95,0.95,0.95}
\definecolor{purple}{rgb}{0.5,0,0.5}
\definecolor{darkgreen}{rgb}{0,0.6,0}
\definecolor{darkblue}{rgb}{0,0,0.6}
\definecolor{brickred}{rgb}{0.7,0.1,0.1}
% Define listings style for LaTeX code
\lstdefinestyle{latexstyle}{
  backgroundcolor=\color{ltgray},
  basicstyle=\ttfamily\small,
  keywordstyle=\color{purple},
  commentstyle=\color{darkgreen}\itshape,
  stringstyle=\color{brickred},
  identifierstyle=\color{black},
  showstringspaces=false,
  frame=single,
  rulecolor=\color{black},
  breaklines=true,
  language=[LaTeX]TeX,
  morekeywords={VoltageSource, circuitikz}, % add your macro names here
  moredelim=**[is][\color{darkblue}]{@}{@}, % for custom highlighting
}

\setlist{noitemsep}
\setlength{\parindent}{0pt}

\title{User Guide for \texttt{circuit.sty}}
\author{Luca Giaccone}
\date{\today}

\begin{document}

\maketitle

\tableofcontents
\newpage

\section{Introduction}

This document provides a complete user guide for the \texttt{circuit.sty} package, authored by Luca Giaccone. The package extends \texttt{circuitikz} by introducing macros that simplify the drawing of common DC, AC, and three-phase circuit components.

The only external dependencies are:
\begin{itemize}
    \item \texttt{circuitikz}
    \item \texttt{xstring}
\end{itemize}

\section{General Notes}

\begin{itemize}
    \item All coordinate pairs (e.g., \texttt{\{0,0\}}) must be passed \textbf{without parentheses}.
    \item Optional arguments may be used to add labels, set orientation, or tweak layout.
    \item The macros are built on top of \texttt{circuitikz} and follow its drawing conventions.
\end{itemize}

%---------------------
\section{Commands Overview and Examples}
%---------------------
\subsection{\texttt{\textbackslash VoltageSource}}
%---------------------

\textbf{Signature:}
\begin{verbatim}
\VoltageSource[side, terminal][<circuitikz options>]{<start>}{<end>}
\end{verbatim}

\textbf{Example 1:}

\begin{lstlisting}[style=latexstyle]
    \begin{circuitikz}[scale=1]
        \VoltageSource[side1, terminal1][l=$E_0$]{0,0}{0,3}
        \VoltageSource[side1, terminal2][l=$E_1$, *-*]{2,0}{2,3}
        \VoltageSource[side2, terminal2][l_=$E_2$, o-o]{4,0}{4,3}
        \VoltageSource[side2, terminal1][l_=$E_3$, i=$I_3$]{6,0}{6,3}
    \end{circuitikz}
\end{lstlisting}

\begin{center}
    \begin{circuitikz}
            \VoltageSource[side1, terminal1][l=$E_0$]{0,0}{0,3}
            \VoltageSource[side1, terminal2][l=$E_1$, *-*]{2,0}{2,3}
            \VoltageSource[side2, terminal2][l_=$E_2$, o-o]{4,0}{4,3}
            \VoltageSource[side2, terminal1][l_=$E_3$, i=$I_3$]{6,0}{6,3}
    \end{circuitikz}
\end{center}

\textbf{Example 2:}

\begin{lstlisting}[style=latexstyle]
    \begin{circuitikz}[scale=1]
        \VoltageSource[][*-o]{0,0}{2,2} % equivalent to side1, terminal1
        \VoltageSource[side1, terminal2][l=$E_4$, -o]{0,0}{-2,2}
        \VoltageSource[side2, terminal2][l=$E_5$, -o]{0,0}{-2,-2}
        \VoltageSource[side2, terminal1][-o]{0,0}{2,-2}
    \end{circuitikz}
\end{lstlisting}

\begin{center}
    \begin{circuitikz}[scale=1]
        \VoltageSource[][*-o]{0,0}{2,2} % equivalent to side1, terminal1
        \VoltageSource[side1, terminal2][l=$E_4$, -o]{0,0}{-2,2}
        \VoltageSource[side2, terminal2][l=$E_5$, -o]{0,0}{-2,-2}
        \VoltageSource[side2, terminal1][-o]{0,0}{2,-2}
    \end{circuitikz}
\end{center}

%---------------------
\subsection{\texttt{\textbackslash CurrentSource}}
%---------------------

\textbf{Signature:}
\begin{verbatim}
\CurrentSource[side, terminal, label][<options>]{<start>}{<end>}
\end{verbatim}

\textbf{Example 1:}

\begin{lstlisting}[style=latexstyle]
    \begin{circuitikz}[scale=1]
        \CurrentSource[side1, terminal1][$A_1$]{0,0}{0,3}
        \CurrentSource[side1, terminal2][$A_2$][*-*]{2.5,0}{2.5,3}
        \CurrentSource[side2, terminal2][$A_3$][o-o]{5,0}{5,3}
        \CurrentSource[side2, terminal1][$A_4$][v^=$V_4$]{7.5,0}{7.5,3}
    \end{circuitikz}
\end{lstlisting}


\begin{center}
    \begin{circuitikz}[scale=1]
        \CurrentSource[side1, terminal1][$A_1$]{0,0}{0,3}
        \CurrentSource[side1, terminal2][$A_2$][*-*]{2.5,0}{2.5,3}
        \CurrentSource[side2, terminal2][$A_3$][o-o]{5,0}{5,3}
        \CurrentSource[side2, terminal1][$A_4$][v^=$V_4$]{7.5,0}{7.5,3}
    \end{circuitikz}
\end{center}

\textbf{Example 2:}

\begin{lstlisting}[style=latexstyle]
    \begin{circuitikz}[scale=1]
        \CurrentSource[side1, terminal1][$A_5$][]{0,0}{2,2}
        \CurrentSource[side1, terminal2][$A_6$][*-*]{0,0}{-2,2}
        \CurrentSource[side2, terminal2][$A_7$][o-o]{0,0}{-2,-2}
        \CurrentSource[side2, terminal1][$A_8$][v^>=$V_8$]{0,0}{2,-2}
    \end{circuitikz}
\end{lstlisting}

\begin{center}
    \begin{circuitikz}[scale=1]
        \CurrentSource[side1, terminal1][$A_5$][]{0,0}{2,2}
        \CurrentSource[side1, terminal2][$A_6$][*-*]{0,0}{-2,2}
        \CurrentSource[side2, terminal2][$A_7$][o-o]{0,0}{-2,-2}
        \CurrentSource[side2, terminal1][$A_8$][v^>=$V_8$]{0,0}{2,-2}
    \end{circuitikz}
\end{center}

%---------------------
\subsection{\texttt{\textbackslash Resistor}}

\textbf{Signature:}
\begin{verbatim}
\Resistor[<options>]{<start>}{<end>}
\end{verbatim}

\textbf{Example 1:}

\begin{lstlisting}[style=latexstyle]
\begin{circuitikz}[scale=1]
	\Resistor[l=$R_1$]{0,0}{0,3}
	\Resistor[l=$R_2$,*-*]{2.5,0}{2.5,3}
	\Resistor[l=$R_3$,o-o]{5,0}{5,3}
	\Resistor[l_=$R_4$,v^=$V_4$]{7.5,0}{7.5,3}
\end{circuitikz}
\end{lstlisting}


\begin{center}
\begin{circuitikz}[scale=1]
	\Resistor[l=$R_1$]{0,0}{0,3}
	\Resistor[l=$R_2$,*-*]{2.5,0}{2.5,3}
	\Resistor[l=$R_3$,o-o]{5,0}{5,3}
	\Resistor[l_=$R_4$,v^=$V_4$]{7.5,0}{7.5,3}
\end{circuitikz}
\end{center}

\textbf{Example 2:}

\begin{lstlisting}[style=latexstyle]
\begin{circuitikz}[scale=1]
	\Resistor[l=$R_5$]{0,0}{2,2}
	\Resistor[l=$R_6$,*-*]{0,0}{-2,2}
	\Resistor[l=$R_7$,o-o]{0,0}{-2,-2}
	\Resistor[l_=$R_8$,v^>=$V_8$]{0,0}{2,-2}
\end{circuitikz}
\end{lstlisting}

\begin{center}
\begin{circuitikz}[scale=1]
	\Resistor[l=$R_5$]{0,0}{2,2}
	\Resistor[l=$R_6$,*-*]{0,0}{-2,2}
	\Resistor[l=$R_7$,o-o]{0,0}{-2,-2}
	\Resistor[l_=$R_8$,v^>=$V_8$]{0,0}{2,-2}
\end{circuitikz}
\end{center}

%---------------------
\subsection{\texttt{\textbackslash Inductor}}

\textbf{Signature:}
\begin{verbatim}
\Inductor[<options>]{<start>}{<end>}
\end{verbatim}

\textbf{Example 1:}

\begin{lstlisting}[style=latexstyle]
\begin{circuitikz}[scale=1]
	\Inductor[l=$L_1$]{0,0}{0,3}
	\Inductor[l=$L_2$,*-*]{2.5,0}{2.5,3}
	\Inductor[l=$L_3$,o-o]{5,0}{5,3}
	\Inductor[l_=$L_4$,v^=$V_4$]{7.5,0}{7.5,3}
\end{circuitikz}
\end{lstlisting}


\begin{center}
\begin{circuitikz}[scale=1]
	\Inductor[l=$L_1$]{0,0}{0,3}
	\Inductor[l=$L_2$,*-*]{2.5,0}{2.5,3}
	\Inductor[l=$L_3$,o-o]{5,0}{5,3}
	\Inductor[l_=$L_4$,v^=$V_4$]{7.5,0}{7.5,3}
\end{circuitikz}
\end{center}

\textbf{Example 2:}

\begin{lstlisting}[style=latexstyle]
\begin{circuitikz}[scale=1]
	\Inductor[l=$L_5$]{0,0}{2,2}
	\Inductor[l=$L_6$,*-*]{0,0}{-2,2}
	\Inductor[l=$L_7$,o-o]{0,0}{-2,-2}
	\Inductor[l_=$L_8$,v^>=$V_8$]{0,0}{2,-2}
\end{circuitikz}
\end{lstlisting}

\begin{center}
\begin{circuitikz}[scale=1]
	\Inductor[l=$L_5$]{0,0}{2,2}
	\Inductor[l=$L_6$,*-*]{0,0}{-2,2}
	\Inductor[l=$L_7$,o-o]{0,0}{-2,-2}
	\Inductor[l_=$L_8$,v^>=$V_8$]{0,0}{2,-2}
\end{circuitikz}
\end{center}

%---------------------
\subsection{\texttt{\textbackslash Capacitor}}

\textbf{Signature:}
\begin{verbatim}
\Capacitor[<options>]{<start>}{<end>}
\end{verbatim}

\textbf{Example 1:}

\begin{lstlisting}[style=latexstyle]
\begin{circuitikz}[scale=1]
	\Capacitor[l=$C_1$]{0,0}{0,3}
	\Capacitor[l=$C_2$,*-*]{2.5,0}{2.5,3}
	\Capacitor[l=$C_3$,o-o]{5,0}{5,3}
	\Capacitor[l_=$C_4$,v^=$V_4$]{7.5,0}{7.5,3}
\end{circuitikz}
\end{lstlisting}


\begin{center}
\begin{circuitikz}[scale=1]
	\Capacitor[l=$C_1$]{0,0}{0,3}
	\Capacitor[l=$C_2$,*-*]{2.5,0}{2.5,3}
	\Capacitor[l=$C_3$,o-o]{5,0}{5,3}
	\Capacitor[l_=$C_4$,v^=$V_4$]{7.5,0}{7.5,3}
\end{circuitikz}
\end{center}

\textbf{Example 2:}

\begin{lstlisting}[style=latexstyle]
\begin{circuitikz}[scale=1]
	\Capacitor[l=$C_5$]{0,0}{2,2}
	\Capacitor[l=$C_6$,*-*]{0,0}{-2,2}
	\Capacitor[l=$C_7$,o-o]{0,0}{-2,-2}
	\Capacitor[l_=$C_8$,v^>=$V_8$]{0,0}{2,-2}
\end{circuitikz}
\end{lstlisting}

\begin{center}
\begin{circuitikz}[scale=1]
	\Capacitor[l=$C_5$]{0,0}{2,2}
	\Capacitor[l=$C_6$,*-*]{0,0}{-2,2}
	\Capacitor[l=$C_7$,o-o]{0,0}{-2,-2}
	\Capacitor[l_=$C_8$,v^>=$V_8$]{0,0}{2,-2}
\end{circuitikz}
\end{center}

%---------------------
\subsection{\texttt{\textbackslash Impedance}}

\textbf{Signature:}
\begin{verbatim}
\Impedance[<options>]{<start>}{<end>}
\end{verbatim}

\textbf{Example 1:}

\begin{lstlisting}[style=latexstyle]
\begin{circuitikz}[scale=1]
	\Impedance[l=$\bar{Z}_1$]{0,0}{0,3}
	\Impedance[l=$\bar{Z}_2$,*-*]{2.5,0}{2.5,3}
	\Impedance[l=$\bar{Z}_3$,o-o]{5,0}{5,3}
	\Impedance[l_=$\bar{Z}_4$,v^=$V_4$]{7.5,0}{7.5,3}
\end{circuitikz}
\end{lstlisting}


\begin{center}
\begin{circuitikz}[scale=1]
	\Impedance[l=$\bar{Z}_1$]{0,0}{0,3}
	\Impedance[l=$\bar{Z}_2$,*-*]{2.5,0}{2.5,3}
	\Impedance[l=$\bar{Z}_3$,o-o]{5,0}{5,3}
	\Impedance[l_=$\bar{Z}_4$,v^=$V_4$]{7.5,0}{7.5,3}
\end{circuitikz}
\end{center}

\textbf{Example 2:}

\begin{lstlisting}[style=latexstyle]
\begin{circuitikz}[scale=1]
	\Impedance[l=$\bar{Z}_5$]{0,0}{2,2}
	\Impedance[l=$\bar{Z}_6$,*-*]{0,0}{-2,2}
	\Impedance[l=$\bar{Z}_7$,o-o]{0,0}{-2,-2}
	\Impedance[l_=$\bar{Z}_8$,v^>=$V_8$]{0,0}{2,-2}
\end{circuitikz}
\end{lstlisting}

\begin{center}
\begin{circuitikz}[scale=1]
	\Impedance[l=$\bar{Z}_5$]{0,0}{2,2}
	\Impedance[l=$\bar{Z}_6$,*-*]{0,0}{-2,2}
	\Impedance[l=$\bar{Z}_7$,o-o]{0,0}{-2,-2}
	\Impedance[l_=$\bar{Z}_8$,v^>=$V_8$]{0,0}{2,-2}
\end{circuitikz}
\end{center}

%---------------------
\subsection{\texttt{\textbackslash Short}}

\textbf{Signature:}
\begin{verbatim}
\Short[<options>]{<start>}{<end>}
\end{verbatim}

\textbf{Example 1:}

\begin{lstlisting}[style=latexstyle]
\begin{circuitikz}[scale=1]
	\Short[i=$I_{cc1}$]{0,0}{0,3}
	\Short[i<=$I_{cc2}$,*-*]{2.5,0}{2.5,3}
	\Short[i_=$I_{cc3}$, o-o]{5,0}{5,3}
	\Short[i_<=$I_{cc4}$]{7.5,0}{7.5,3}
\end{circuitikz}
\end{lstlisting}


\begin{center}
\begin{circuitikz}[scale=1]
	\Short[i=$I_{cc1}$]{0,0}{0,3}
	\Short[i<=$I_{cc2}$,*-*]{2.5,0}{2.5,3}
	\Short[i_=$I_{cc3}$, o-o]{5,0}{5,3}
	\Short[i_<=$I_{cc4}$]{7.5,0}{7.5,3}
\end{circuitikz}
\end{center}

\textbf{Example 2:}

\begin{lstlisting}[style=latexstyle]
\begin{circuitikz}[scale=1]
	\Short[i=$I_{cc5}$, *-o]{0,0}{2,2}
	\Short[i<=$I_{cc6}$, -o]{0,0}{-2,2}
	\Short[i_=$I_{cc7}$, -o]{0,0}{-2,-2}
	\Short[i_<=$I_{cc8}$, -o]{0,0}{2,-2}
\end{circuitikz}
\end{lstlisting}

\begin{center}
\begin{circuitikz}[scale=1]
	\Short[i=$I_{cc5}$, *-o]{0,0}{2,2}
	\Short[i<=$I_{cc6}$, -o]{0,0}{-2,2}
	\Short[i_=$I_{cc7}$, -o]{0,0}{-2,-2}
	\Short[i_<=$I_{cc8}$, -o]{0,0}{2,-2}
\end{circuitikz}
\end{center}

%---------------------
\subsection{\texttt{\textbackslash Open}}

\textbf{Signature:}
\begin{verbatim}
\Open[<options>]{<start>}{<end>}
\end{verbatim}

\textbf{Example 1:}

\begin{lstlisting}[style=latexstyle]
\begin{circuitikz}[scale=1]
	\Open[v=$V_{oc1}$, *-*]{0,0}{0,3}
	\Open[v_=$V_{oc2}$, o-o]{2.5,0}{2.5,3}
	\Open[v<=$V_{oc3}$, *-*]{5,0}{5,3}
	\Open[v_<=$V_{oc4}$, o-o]{7.5,0}{7.5,3}
\end{circuitikz}
\end{lstlisting}


\begin{center}
\begin{circuitikz}[scale=1]
	\Open[v=$V_{oc1}$, *-*]{0,0}{0,3}
	\Open[v_=$V_{oc2}$, o-o]{2.5,0}{2.5,3}
	\Open[v<=$V_{oc3}$, *-*]{5,0}{5,3}
	\Open[v_<=$V_{oc4}$, o-o]{7.5,0}{7.5,3}
\end{circuitikz}
\end{center}

\textbf{Example 2:}

\begin{lstlisting}[style=latexstyle]
\begin{circuitikz}[scale=1]
	\Open[v=$V_{oc5}$, *-*]{0,0}{2,2}
	\Open[v_=$V_{oc6}$, o-o]{0,0}{-2,2}
	\Open[v<=$V_{oc7}$, *-*]{0,0}{-2,-2}
	\Open[v_<=$V_{oc8}$, o-o]{0,0}{2,-2}
\end{circuitikz}
\end{lstlisting}

\begin{center}
\begin{circuitikz}[scale=1]
	\Open[v=$V_{oc5}$, *-*]{0,0}{2,2}
	\Open[v_=$V_{oc6}$, o-o]{0,0}{-2,2}
	\Open[v<=$V_{oc7}$, *-*]{0,0}{-2,-2}
	\Open[v_<=$V_{oc8}$, o-o]{0,0}{2,-2}
\end{circuitikz}
\end{center}

%---------------------
\subsection{\texttt{\textbackslash Switch}}

\textbf{Signature:}
\begin{verbatim}
\Switch[opening|closing][<options>]{<start>}{<end>}
\end{verbatim}

\textbf{Example 1:}

\begin{lstlisting}[style=latexstyle]
\begin{circuitikz}[scale=1]
	\Switch[opening][l={$t=0$}]{0,0}{0,3}
	\Switch[opening][l={$t=t_0$}, *-*]{2.5,0}{2.5,3}
	\Switch[closing][l={$t=t_1$}, o-o]{5,0}{5,3}
	\Switch[closing][l={$t=t_2$}, ,v=$V_S$]{7.5,0}{7.5,3}
\end{circuitikz}
\end{lstlisting}


\begin{center}
\begin{circuitikz}[scale=1]
	\Switch[opening][l={$t=0$}]{0,0}{0,3}
	\Switch[opening][l={$t=t_0$}, *-*]{2.5,0}{2.5,3}
	\Switch[closing][l={$t=t_1$}, o-o]{5,0}{5,3}
	\Switch[closing][l={$t=t_2$}, ,v=$V_S$]{7.5,0}{7.5,3}
\end{circuitikz}
\end{center}

\textbf{Example 2:}

\begin{lstlisting}[style=latexstyle]
\begin{circuitikz}[scale=1]
	\Switch[opening][l={$t=0$}]{0,0}{2,2}
	\Switch[opening][l={$t=t_0$}, *-*]{0,0}{-2,2}
	\Switch[closing][l={$t=t_1$}, o-o]{0,0}{-2,-2}
	\Switch[closing][l={$t=t_2$}, ,v=$V_S$]{0,0}{2,-2}
\end{circuitikz}
\end{lstlisting}

\begin{center}
\begin{circuitikz}[scale=1]
	\Switch[opening][l={$t=0$}]{0,0}{2,2}
	\Switch[opening][l={$t=t_0$}, *-*]{0,0}{-2,2}
	\Switch[closing][l={$t=t_1$}, o-o]{0,0}{-2,-2}
	\Switch[closing][l={$t=t_2$}, ,v=$V_S$]{0,0}{2,-2}
\end{circuitikz}
\end{center}

% \newpage
%---------------------
\subsection{\texttt{\textbackslash ThreePhaseGen}}

\textbf{Signature:}
\begin{verbatim}
\ThreePhaseGen[horizontal|vertical][<length>]{<origin>}
\end{verbatim}

\textbf{Example:}

\begin{lstlisting}[style=latexstyle]
\begin{circuitikz}
    \ThreePhaseGen[horizontal][3]{0,0}
    \ThreePhaseGen[vertical][2]{8,-1.5}
\end{circuitikz}
\end{lstlisting}

\begin{center}
    \begin{circuitikz}
        \ThreePhaseGen[horizontal][3]{0,0}
        \ThreePhaseGen[vertical][2]{8,-1.5}
    \end{circuitikz}
\end{center}

% %---------------------
\subsection{\texttt{\textbackslash ThreePhaseLoad}}

\textbf{Signature:}
\begin{verbatim}
\ThreePhaseLoad[vertical|horizontal][<label1>][<label2>]{<origin>}
\end{verbatim}

\textbf{Examples:}

\begin{lstlisting}[style=latexstyle]
\begin{circuitikz}
    \ThreePhaseLoad[vertical][$P$][$Q$]{0,0}
    \ThreePhaseLoad[horizontal][$Q_C$]{8,0}
\end{circuitikz}
\end{lstlisting}

\begin{center}
\begin{circuitikz}
    \ThreePhaseLoad[vertical][$P$][$Q$]{0,0}
    \ThreePhaseLoad[horizontal][$Q_C$]{8,0}
\end{circuitikz}
\end{center}



%---------------------
\subsection{\texttt{\textbackslash ThreePhaseLine}}

\textbf{Signature:}
\begin{verbatim}
\ThreePhaseLine[horizontal|vertical, type][<length>][<label>]{<origin>}
\end{verbatim}

\textbf{Example:}

\begin{lstlisting}[style=latexstyle]
\begin{circuitikz}
    \ThreePhaseLine[horizontal, impedance][3][$\bar{Z}_L$]{0,0}
    \ThreePhaseLine[vertical, resistor][3][$R$]{6.5,-1.5}
    \ThreePhaseLine[vertical, inductor][3][$L$]{12,-1.5}
\end{circuitikz}
\end{lstlisting}

\begin{center}
\begin{circuitikz}
    \ThreePhaseLine[horizontal, impedance][3][$\bar{Z}_L$]{0,0}
    \ThreePhaseLine[vertical, resistor][3][$R$]{6.5,-1.5}
    \ThreePhaseLine[vertical, inductor][3][$L$]{12,-1.5}
\end{circuitikz}
\end{center}



\newpage
\section{Example Circuits}

\subsection{DC Circuit}

\begin{lstlisting}[style=latexstyle]
\begin{circuitikz}[scale=1]
    \VoltageSource[side1, terminal2][l=$E_1$]{0,0}{0,3}
    \Resistor[l=$R_1$]{0,3}{3,3}
    \Resistor[l=$R_2$]{0,0}{3,0}
    \CurrentSource[side1, terminal2][$A_1$][*-*]{3,0}{3,3}
    \Resistor[l=$R_3$, -*]{3,3}{6,0}
    \Resistor[l=$R_4$, -*]{3,3}{6,3}
    \Switch[closing][l={t=0}]{6,3}{9,3}
    \Capacitor[l=$C_1$, v<=$v(t)$]{9,3}{9,0}
    \draw (3,0) -- (6,0) -- (6,3);
    \draw (6,0) -- (9,0);
\end{circuitikz}
\end{lstlisting}

\begin{center}
	\begin{circuitikz}[scale=1]
		\VoltageSource[side1, terminal2][l=$E_1$]{0,0}{0,3}
		\Resistor[l=$R_1$]{0,3}{3,3}
		\Resistor[l=$R_2$]{0,0}{3,0}
		\CurrentSource[side1, terminal2][$A_1$][*-*]{3,0}{3,3}
		\Resistor[l=$R_3$, -*]{3,3}{6,0}
		\Resistor[l=$R_4$, -*]{3,3}{6,3}
		\Switch[closing][l={t=0}]{6,3}{9,3}
		\Capacitor[l=$C_1$, v<=$v(t)$]{9,3}{9,0}
		\draw (3,0) -- (6,0) -- (6,3);
		\draw (6,0) -- (9,0);
	\end{circuitikz}
\end{center}

\subsection{AC Circuit}

\begin{lstlisting}[style=latexstyle]
\begin{circuitikz}[scale=1]
    \VoltageSource[side1, terminal2][l=$\bar{E}_1$]{0,0}{0,6}
    \CurrentSource[side1, terminal1][$\bar{A}_1$][*-*]{3,3}{6,3}
    \Impedance[l=$\bar{Z}_1$, *-, v<=$\bar{V}_1$]{3,6}{3,3}
    \Impedance[l=$\bar{Z}_2$, -*, i=$\bar{I}_2$]{3,3}{3,0}
    \Impedance[l=$\bar{Z}_3$, i^<=$\bar{I}_3$]{6,6}{6,3}
    \Impedance[l=$\bar{Z}_4$,  v<=$\bar{V}_4$]{6,3}{6,0}
    \draw (0,0) -- (6,0);
    \draw (0,6) -- (6,6);
\end{circuitikz} 
\end{lstlisting}

\begin{center}
	\begin{circuitikz}[scale=1]
		\VoltageSource[side1, terminal2][l=$\bar{E}_1$]{0,0}{0,6}
		\CurrentSource[side1, terminal1][$\bar{A}_1$][*-*]{3,3}{6,3}
		\Impedance[l=$\bar{Z}_1$, *-, v<=$\bar{V}_1$]{3,6}{3,3}
		\Impedance[l=$\bar{Z}_2$, -*, i=$\bar{I}_2$]{3,3}{3,0}
		\Impedance[l=$\bar{Z}_3$, i^<=$\bar{I}_3$]{6,6}{6,3}
		\Impedance[l=$\bar{Z}_4$,  v<=$\bar{V}_4$]{6,3}{6,0}
		\draw (0,0) -- (6,0);
		\draw (0,6) -- (6,6);
	\end{circuitikz}   
\end{center}

\subsection{Three-Phase Circuit}

\begin{lstlisting}[style=latexstyle]
\begin{circuitikz}[]%,line width=0.5pt]
    
        % generator and line
        \ThreePhaseGen[horizontal][2]{0,4}
        \ThreePhaseLine[horizontal, impedance][2.5][$\bar{Z}_L$]{2,4}
    
        % load A
        \Short{4.5, 5.5}{9, 5.5};
        \Short{4.5, 4}{9, 4};
        \Short{4.5, 2.5}{9, 2.5};
        \ThreePhaseLoad[vertical][$P_A$][$Q_A$]{10,4}
        
        % load B
        \Short[*-]{5,5.5}{5, 1}
        \Short[*-]{6.5,4}{6.5, 1}
        \Short[*-]{8,2.5}{8, 1}
        \ThreePhaseLoad[horizontal][$P_B$][$\cos \varphi_B$]{6.5,0}
    
        % current and voltages defined by means of Short and Open
        \Open[i=$\bar{I}_A$]{6,5.5}{8.5,5.5}
        \Open[i=$\bar{I}_B$]{5,5.5}{5,3}
        \Open[v=$\bar{V}_g$]{2,4}{2,5.5}
        \Open[v^=$\bar{V}$]{8.5,4}{8.5,5.5}
    
    \end{circuitikz}
\end{lstlisting}

\begin{center}
   
    \begin{circuitikz}[]%,line width=0.5pt]
    
        % generator and line
        \ThreePhaseGen[horizontal][2]{0,4}
        \ThreePhaseLine[horizontal, impedance][2.5][$\bar{Z}_L$]{2,4}
    
        % load A
        \Short{4.5, 5.5}{9, 5.5};
        \Short{4.5, 4}{9, 4};
        \Short{4.5, 2.5}{9, 2.5};
        \ThreePhaseLoad[vertical][$P_A$][$Q_A$]{10,4}
        
        % load B
        \Short[*-]{5,5.5}{5, 1}
        \Short[*-]{6.5,4}{6.5, 1}
        \Short[*-]{8,2.5}{8, 1}
        \ThreePhaseLoad[horizontal][$P_B$][$\cos \varphi_B$]{6.5,0}
    
        % current and voltages defined by means of Short and Open
        \Open[i=$\bar{I}_A$]{6,5.5}{8.5,5.5}
        \Open[i=$\bar{I}_B$]{5,5.5}{5,3}
        \Open[v=$\bar{V}_g$]{2,4}{2,5.5}
        \Open[v^=$\bar{V}$]{8.5,4}{8.5,5.5}
    
    \end{circuitikz}
\end{center}

\section{Final Notes}

Ensure that the \texttt{circuit.sty} file is in your local project directory or in your \texttt{TEXMFHOME} tree. This guide provides all needed information and examples for typical usage of the macros defined in the package.

\end{document}
